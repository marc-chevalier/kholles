%&LaTeX
\documentclass[a4paper,10pt]{article}%||article (twoside)
\usepackage[utf8]{inputenc}	%Francais
\usepackage[T1]{fontenc}	%Francais
\usepackage[francais]{babel}%Francais
\usepackage{xkeyval}
\usepackage{tikz}
\usepackage{layout}		%gabarit de page
\usepackage{boiboites}	%Mes belles boites
\usepackage{geometry}	%réglage des marges
\usepackage{setspace}	%Interligne
\usepackage{soul}		%Souligne et barre
\usepackage{ulem}		%Souligne
\usepackage{eurosym}	%Symbole €
\usepackage{graphicx}	%Images
%\usepackage{bookman}	%Police
%\usepackage{charter}	%Police
%\usepackage{newcent}	%Police
%\usepackage{lmodern}	%Police
\usepackage{mathpazo}	%Police
%\usepackage{mathptmx}	%Police
\usepackage{url}		%Citation d'url
\usepackage{verbatim}	%Citation de code (verbatimet verbatimtab
\usepackage{moreverb}	%Citation de code
\usepackage{listings}	%Citation de code coloré
\usepackage{fancyhdr}	%\pagestyle{fancy}
\usepackage{wrapfig}    %Insertion d'une image dans un paragraphe
\usepackage{color}		%Couleurs
\usepackage{colortbl}	%Couleurs dans un tableau
\usepackage{amsmath}	%Maths
\usepackage{amssymb}	%Maths
\usepackage{mathrsfs}	%Maths
\usepackage{amsthm}		%Maths
\usepackage{makeidx}	%Création d'index
%\usepackage{thmbox} 	%Boites en potence
\usepackage{stmaryrd}	%Toutes sortes de symboles bizarres
\usepackage{listings}	%Code
\usepackage{subfig}  	%Sous-figures flottantes
\usepackage{placeins}	%\FloatBarrier
\usepackage{tikz}
\usepackage{minted}
\usepackage[hidelinks]{hyperref}
\usepackage{slashbox}
\usepackage{xifthen}
\usepackage{longtable}
\usepackage{multirow}

\input xy
\xyoption{all}

%\newtheorem{lemme}{Lemme}
%\newtheorem{theoreme}{Th\'{e}or\`{e}me}
%\newtheorem[L]{thm}{Théorème}[section]
%\newtheorem{proposition}{Proposition}
%\newtheorem{corollaire}{Corollaire}
%\newtheorem{definition}{D\'{e}finition}
%\newtheorem{notation}{Notation}
%\newtheorem{remarque}{Remarque}

\newboxedtheorem[boxcolor=red, background=red!5, titlebackground=red!50,titleboxcolor = black]{theoreme}{Th\'{e}or\`{e}me}{TheC}		%Theoreme
\newboxedtheorem[boxcolor=orange, background=orange!5, titlebackground=orange!50,titleboxcolor = black]{definition}{D\'{e}finition}{DefC}			%Definition
\newboxedtheorem[boxcolor=blue, background=blue!5, titlebackground=blue!20,titleboxcolor = black]{proposition}{Proposition}{ProC}				%Proposition
\newboxedtheorem[boxcolor=cyan, background=cyan!5, titlebackground=cyan!20,titleboxcolor = black]{corollaire}{Corollaire}{CorC}				%Corollaire
\newboxedtheorem[boxcolor=blue, background=blue!5, titlebackground=blue!20,titleboxcolor = black]{remarque}{Remarque}{RemC}				%Remarque
\newboxedtheorem[boxcolor=green, background=green!5, titlebackground=green!30,titleboxcolor = black]{notation}{Notation}{NotC}				%Notation
\newboxedtheorem[boxcolor=yellow, background=yellow!0, titlebackground=yellow!30,titleboxcolor = black]{exemple}{Exemple}{ExeC}					%Exemple
\newboxedtheorem[boxcolor=magenta, background=magenta!5, titlebackground=magenta!30,titleboxcolor = black]{lemme}{Lemme}{LemC}					%Lemme


%Ensembles usuels
\renewcommand{\AA}{\mathbb{A}}
\newcommand{\CC}{\mathbb{C}}
\newcommand{\HH}{\mathbb{H}}
\newcommand{\KK}{\mathbb{K}}
\newcommand{\MM}{\mathbb{M}}
\newcommand{\NN}{\mathbb{N}}
\newcommand{\OO}{\mathbb{O}}
\newcommand{\PP}{\mathbb{P}}
\newcommand{\QQ}{\mathbb{Q}}
\newcommand{\RR}{\mathbb{R}}
\newcommand{\TT}{\mathbb{T}}
\newcommand{\ZZ}{\mathbb{Z}}

%Lettres rondes
\newcommand{\A}{\mathcal{A}}
\newcommand{\B}{\mathcal{B}}
\newcommand{\C}{\mathcal{C}}
\newcommand{\D}{\mathcal{D}}
\newcommand{\E}{\mathcal{E}}
\newcommand{\F}{\mathcal{F}}
\newcommand{\K}{\mathcal{K}}
\renewcommand{\L}{\mathcal{L}}
\newcommand{\M}{\mathcal{M}}
\newcommand{\N}{\mathcal{N}}
\renewcommand{\P}{\mathcal{P}}
\newcommand{\R}{\mathcal{R}}
\renewcommand{\S}{\mathcal{S}}
\newcommand{\T}{\mathcal{T}}
\newcommand{\V}{\mathcal{V}}
\newcommand{\W}{\mathcal{W}}
\newcommand{\X}{\mathcal{X}}

%Fleches
\newcommand{\ra}{\rightarrow}
\newcommand{\la}{\leftarrow}
\newcommand{\iso}{\stackrel{\sim}{\ra}}
\newcommand{\giso}{\stackrel{\sim}{\la}}
\newcommand{\isononcan}{\simeq}
\newcommand{\equ}{\approx}
\newcommand{\Ra}{\Rightarrow}
\newcommand{\longra}{\longrightarrow}
\newcommand{\longRa}{\Longrightarrow}
\newcommand{\La}{\Leftarrow}
\newcommand{\longla}{\longleftarrow}
\newcommand{\longLa}{\Longleftarrow}
\newcommand{\lra}{\leftrightarrow}
\newcommand{\LRa}{\Leftrightarrow}
\newcommand{\longLRa}{\Longleftrightarrow}

%Texte
\newcommand{\ssi}{si et seulement si\xspace}
\newcommand{\cad}{c'est-\`a-dire\xspace}
\newcommand{\num}{{$\mathrm{n}^{\mathrm{o}}$}}
\renewcommand{\thefootnote}{\arabic{footnote}}
\newcommand{\Turing}{\textsc{Turing }}
\newcommand{\Grzeg}{\textsc{Grzegorczyk}}
\newcommand{\Max}{\mathrm{Max}}
\renewcommand{\Pr}{\mathrm{Pr}}

%Symboles
\newcommand{\argmin}{\text{argmin}}
\newcommand{\rainbowdash}{\vdash}
\newcommand{\notrainbowdash}{\nvdash}
\newcommand{\rainbowDash}{\vDash}
\newcommand{\notrainbowDash}{\nvDash}
\newcommand{\Rainbowdash}{\Vdash}
\newcommand{\notRainbowdash}{\nVdash}
\newcommand{\bottom}{\bot}



%Du Chevalier
\newcommand{\set}[2]
{
    \ifthenelse{\equal{#2}{}}
    {
        \left\lbrace #1 \right\rbrace
    }{
        \left\lbrace #1 \:\left\vert\: #2 \vphantom{#1} \right.\right\rbrace
    }
}
\newcommand{\unionDisjointe}{\sqcup}
\newcommand{\soustractionNaturelle}{\dot{-}}
\newcommand{\iverson}[1]{\left[ #1\right]}
\newcommand{\functionArrow}{\rightarrow}
\newcommand{\surj}{\twoheadrightarrow}
\newcommand{\inj}{\hookrightarrow}
\newcommand{\bij}{\hookrightarrow \hspace{-10pt} \rightarrow}
\newcommand{\composition}{\circ}
\newcommand{\leftTuring}{\longleftarrow}
\newcommand{\rightTuring}{\longrightarrow}
\newcommand{\HALT}{\texttt{HALT}}
\newcommand{\nullTuring}{\bigcirc}
\newcommand{\UTMInit}{\diamondsuit}
\newcommand{\UTMFin}{\ddagger}
\newcommand{\kleene}[1]{#1^\star}
\newcommand{\vect}[2]
{
    \ifthenelse{\equal{#2}{}}
    {
        \mathrm{Vect}\left(#1\right)
    }{
        \mathrm{Vect}_{#1}\left(#2\right)
    }
}
\newcommand{\norme}[2]
{
    \ifthenelse{\equal{#2}{}}
    {
        \left\lVert #1 \right\rVert
    }{
        \left\lVert #1 \right\rVert_{#2}
    }
}
\newcommand{\rec}[2]{\mathrm{Rec}\left( #1, #2 \right)}
\newcommand{\RecB}[3]{\mathrm{RecB}\left( #1, #2, #3 \right)}
\newcommand{\SumB}[1]{\mathrm{SumB}\left( #1 \right)}
\newcommand{\ProdB}[1]{\mathrm{ProdB}\left( #1 \right)}
\newcommand{\MinB}[1]{\mathrm{MinB}\left( #1 \right)}
\newcommand{\Min}[1]{\mathrm{Min}\left( #1 \right)}
\newcommand{\nonDefini}{\bot}
\newcommand{\hyper}[1]{H_{#1}}
\newcommand{\knuth}[1]
{
    \ifthenelse{\equal{#1}{1}}
    {
        \uparrow
    }{
        \ifthenelse{\equal{#1}{2}}
        {
            \uparrow\uparrow
        }{
            \uparrow^{#1}
        }
    }
}
\newcommand{\et}{\wedge}
\newcommand{\biget}{\bigwedge}
\newcommand{\ou}{\vee}
\newcommand{\xor}{\oplus}
\newcommand{\non}{\neg}
\newcommand{\congru}{\equiv}
\newcommand{\inclu}{\subseteq}
\newcommand{\domine}{\prec}
\newcommand{\Ackermann}[1]{\mathcal{A}_{#1}}
\newcommand{\AckerDom}[1]{\mathfrak{T}_{#1}}
\newcommand{\alphaeq}{\leftrightarrow_\alpha}
\newcommand{\betared}{\rightarrow_\beta}
\newcommand{\betaeq}{=_\beta}
\newcommand{\etaeq}{\leftrightarrow_\eta}
\newcommand{\freeVars}[1]{freeVars\left(#1\right)}
\newcommand{\subs}[3]{\left[#1/#2\vphantom{#3}\right]#3}
\newcommand{\trans}{\ra}
\newcommand{\rtrans}{\ra}
\newcommand{\lrtrans}{\leftrightarrow}
\newcommand{\motVide}{\varepsilon}
\newcommand{\longueur}[1]{\left\lvert #1 \right\rvert}
\newcommand{\card}[1]{\left\lvert #1 \right\rvert}
\newcommand{\EulerMascheroni}{\gamma}
\newcommand{\eqFun}{\dot{=}}
\newcommand{\entiers}[2]{\left\llbracket #1, #2 \right\rrbracket}
\newcommand{\GrzegFun}[1]{\mathfrak{f}_{#1}}
\newcommand{\fst}{\pi_1}
\newcommand{\snd}{\pi_2}
\newcommand{\Elem}{\E}
\newcommand{\GrzegClass}[1]{\E^{#1}}
\newcommand{\TuringRed}{\leqslant_T}
\newcommand{\TuringEq}{\equiv_T}
\newcommand{\TuringDeg}[1]{\deg\left( #1 \right)}
\newcommand{\TuringSaut}[1]{#1'}
\newcommand{\nTuringSaut}[2]{#1^{\left( #2 \right)}}
\newcommand{\TuringEmpty}{\emptyset}
\newcommand{\MTR}{\mathfrak{M}}
\newcommand{\MTRf}{\mathfrak{M}_f}
\newcommand{\transCharniere}[2]{\mathfrak{C}\left( #1 , #2 \right)}
\newcommand{\nbOx}[1]{\mathfrak{C}\left( #1 \right)}
\newcommand{\dotsVirgule}{\ldots}
\newcommand{\dotsPlus}{\cdots}
\newcommand{\dotsLambda}{\ldots}
\newcommand{\dotsMot}{\ldots}
\newcommand{\dotsTape}{\cdots}
\newcommand{\dotsTuring}{\ldots}
\newcommand{\susp}{...}
\newcommand{\dotsProd}{\cdots}
\newcommand{\floor}[1]{\left\lfloor #1 \right\rfloor}
\newcommand{\ceil}[1]{\left\lceil #1 \right\rceil}
\renewcommand{\angle}[1]{\left\langle #1 \right\rangle}
\newcommand{\abs}[2]{\lambda #1.#2}
\newcommand{\app}[2]{#1\,#2}
\newcommand{\Space}{\texttt{[Space]}}
\newcommand{\LF}{\texttt{[LF]}}
\newcommand{\Tab}{\texttt{[Tab]}}
\newcommand{\parties}[1]{\mathcal{P}\left(#1\right)}
\newcommand{\TuringRedStrict}{<_T}
\newcommand{\Rat}[1]{\mathfrak{R}\left(#1\right)}
\newcommand{\UTMBlanc}{\texttt{\_}}
\newcommand{\dd}{\mathrm{d}}
\newcommand{\grandO}[1]{\mathcal{O}\left(#1\right)}
\newcommand{\petitO}[1]{o\left(#1\right)}
\newcommand{\opname}[1]{\operatorname{#1}}
\newcommand{\Var}[1]{\text{Var}\left( #1 \right)}
\newcommand{\prob}[1]{\PP\left( #1 \right)}
\newcommand{\esp}[1]{\EE\left[ #1 \right]}

\title{Exercices de khôlle}
\author{Marc \textsc{Chevalier}}
\date{16 octobre 2014}

\begin{document}
\maketitle
\setcounter{tocdepth}{2}
\tableofcontents

\section{Séries}

\subsection{DL de la série harmonique}

Prouver :

\[
    \sum_{k=1}^n \frac{1}{k}= \ln(n)+\gamma+\frac{1}{2n}-\frac{1}{12n^2}+\frac{1}{120n^4}-\frac{1}{252n^6}+\frac{1}{240n^8}-\frac{1}{132n^{10}}+\grandO{\frac1{n^{12}}}
\]

\subsection{Théorème de \textsc{Riemann}}

Soit $(u_n)_{n\in\NN}$ une suite à valeurs réelles dont la série associée est semi-convergente, c'est-à-dire que et soit
$\alpha\in\RR\cup\{-\infty,+\infty\}$.

Alors il existe une permutation $\sigma$ de $\NN$ telle que
$$\sum_{k=0}^nu_{\sigma(k)}\underset{n\to\infty}{\longrightarrow}\alpha.$$

\subsection{Arnaud \textsc{Triay}}

$u_n$ CV $\to$ 0 décroissante, montrer que $\sum u_n$ et $\sum n(u_n - u_{n+1})$ sont de même nature et égales en cas de convergence.

\subsection{Arnaud \textsc{Triay}, le retour}

$a, b > 0, u_0 > 0$
$u_{n+1} = \frac{n+a}{n+b}u_n$
\begin{enumerate}
    \item Montrer que $\exists \alpha > 0, u_n \sim \frac{\alpha}{n(b-a)}$.
    \item Trouver une condition nécessaire et suffisante pour que $\sum u_n$ CV.
\end{enumerate}


\subsection{Inégalité de \textsc{Carleman}}

Soit $(a_n)$ une suite à termes positifs tels que $\sum_na_n$ CV.

\begin{enumerate}
    \item Prouver que la série de terme général $\frac{\sum_{i=1}^n ia_i}{n(n+1)}$ CV et est de même somme que la série de terme général $a_n$.
    \item Montrer l'inégalité $\frac{1}{(n!)^{\frac{1}{n}}} \leqslant \frac{e}{n+1}$
    \item En conclure que $$\sum_{n=1}^{+\infty}\left(\prod_{i=1}^n a_i\right)^\frac{1}{n} \leqslant e \sum_{n=1}^{+\infty} a_n$$
\end{enumerate}

\subsection{Somme de logarithmes}

Par comparaison à une intégrale, donner un équivalent de $u_n = \sum_{k=1}^n\ln^2(k)$. La série de terme général $\frac{1}{u_n}$ est-elle convergente ?

\subsection{Somme et développement asymptotique de la série des inverses des carrés}

Le but de l'exercice est de calculer $\sum_{n\geqslant 1}\frac{1}{n^2}$ et de donner un développement asymptotique de la somme partielle $\sum_{k=1}^n\frac{1}{i^2}$.

\begin{enumerate}
    \item   \begin{enumerate}
                \item Soit $\alpha >1$ et $k\geqslant 2$.  Démontrer que
                \[
                    \int_k^{k+1}\frac{\dd t}{t^\alpha} \leqslant \frac{1}{k^\alpha} \leqslant \int_{k-1}^k \frac{\dd t}{t^\alpha}
                \]
                \item En déduire que 
                \[
                    \sum_{k\geqslant n} \frac{1}{k^\alpha} \underset{+\infty}{\sim} \frac{1}{(\alpha -1)n^{\alpha-1}}
                \]
            \end{enumerate}
    \item Soit $f$ une fonction de classe $\C^1$ sur $[0,\pi]$. Démontrer que
    \[
        \int_0^\pi f(t)\sin\left(\frac{(2n+1)t}{2}\right)\dd t \to 0
    \]
    \item On pose $A_n(t) = \frac{1}{2}+\sum_{k=1}^n \cos(kt)$. Vérifier que, pour tout $t\in]0,\pi]$, on a
    \[
        A_n(t)=\frac{\sin \left( \frac{(2n+1)t}{2}\right)}{2\sin\left(\frac{t}{2}\right)}
    \]
    \item Déterminer deux réels $a$ et $b$ tels que, pour tout $n \geqslant 1$
    \[
        \int_0^\pi (at^2+bt)\cos(nt)\dd t=\frac{1}{n^2}
    \]
    
    Vérifier alors que 
    \[
        \int_0^\pi (at^2+bt)A_n(t) \dd t = S_n-\frac{\pi^2}{6}
    \]
    \item Déduire des questions précédentes que $S_n \to \frac{\pi^2}{6}$
    \item Déduire des questions précédentes que 
    \[
        S_n=\frac{\pi^2}{6}-\frac{1}{n}+\petitO{\frac{1}{n}}
    \]
\end{enumerate}

\subsection{Agathe \textsc{Herrou}}

Nature et somme

\[
    \sum_{n=0}^{\infty} \left(\frac{1}{4n+1} - \frac{2}{4n+2} + \frac{1}{4n+3} + \frac{1}{4n+4}\right)
\]

\subsection{Série harmonique alternée}

Convergence et somme de 
\[
    \sum\limits_{n=0}^\infty \frac{(-1)^n}{n}
\]

\subsection{Formule de \textsc{Stirling}}

Convergence de 
\[
    \sum\frac{3^n(n!)^2}{(2n)!}
\]

\subsection{Convergence du carré de la moyenne}

Soient $(u_n)_{n\geqslant 1}$ une suite réelle. Pour $n \geqslant 1$, on pose $v_n = \frac{u_1 + \cdots + u_n}{n}$. Montrer que si la série de terme général $(u_n)^2$ converge alors la série de terme général $(v_n)^2$ converge et que 
\[
    \sum\limits_{n=1}^{+\infty} (v_n)^2 \leqslant 4 \sum_{n=1}^{+\infty} (u_n)^2
\]


\end{document}