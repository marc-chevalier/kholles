%&LaTeX
\documentclass[a4paper,10pt]{article}%||article (twoside)
\usepackage[utf8]{inputenc}	%Francais
\usepackage[T1]{fontenc}	%Francais
\usepackage[francais]{babel}%Francais
\usepackage{xkeyval}
\usepackage{tikz}
\usepackage{layout}		%gabarit de page
\usepackage{boiboites}	%Mes belles boites
\usepackage{geometry}	%réglage des marges
\usepackage{setspace}	%Interligne
\usepackage{soul}		%Souligne et barre
\usepackage{ulem}		%Souligne
\usepackage{eurosym}	%Symbole €
\usepackage{graphicx}	%Images
%\usepackage{bookman}	%Police
%\usepackage{charter}	%Police
%\usepackage{newcent}	%Police
%\usepackage{lmodern}	%Police
\usepackage{mathpazo}	%Police
%\usepackage{mathptmx}	%Police
\usepackage{url}		%Citation d'url
\usepackage{verbatim}	%Citation de code (verbatimet verbatimtab
\usepackage{moreverb}	%Citation de code
\usepackage{listings}	%Citation de code coloré
\usepackage{fancyhdr}	%\pagestyle{fancy}
\usepackage{wrapfig}    %Insertion d'une image dans un paragraphe
\usepackage{color}		%Couleurs
\usepackage{colortbl}	%Couleurs dans un tableau
\usepackage{amsmath}	%Maths
\usepackage{amssymb}	%Maths
\usepackage{mathrsfs}	%Maths
\usepackage{amsthm}		%Maths
\usepackage{makeidx}	%Création d'index
%\usepackage{thmbox} 	%Boites en potence
\usepackage{stmaryrd}	%Toutes sortes de symboles bizarres
\usepackage{listings}	%Code
\usepackage{subfig}  	%Sous-figures flottantes
\usepackage{placeins}	%\FloatBarrier
\usepackage{tikz}
\usepackage{minted}
\usepackage[hidelinks]{hyperref}
\usepackage{slashbox}
\usepackage{xifthen}
\usepackage{longtable}
\usepackage{multirow}

\input xy
\xyoption{all}

%\newtheorem{lemme}{Lemme}
%\newtheorem{theoreme}{Th\'{e}or\`{e}me}
%\newtheorem[L]{thm}{Théorème}[section]
%\newtheorem{proposition}{Proposition}
%\newtheorem{corollaire}{Corollaire}
%\newtheorem{definition}{D\'{e}finition}
%\newtheorem{notation}{Notation}
%\newtheorem{remarque}{Remarque}

\newboxedtheorem[boxcolor=red, background=red!5, titlebackground=red!50,titleboxcolor = black]{theoreme}{Th\'{e}or\`{e}me}{TheC}		%Theoreme
\newboxedtheorem[boxcolor=orange, background=orange!5, titlebackground=orange!50,titleboxcolor = black]{definition}{D\'{e}finition}{DefC}			%Definition
\newboxedtheorem[boxcolor=blue, background=blue!5, titlebackground=blue!20,titleboxcolor = black]{proposition}{Proposition}{ProC}				%Proposition
\newboxedtheorem[boxcolor=cyan, background=cyan!5, titlebackground=cyan!20,titleboxcolor = black]{corollaire}{Corollaire}{CorC}				%Corollaire
\newboxedtheorem[boxcolor=blue, background=blue!5, titlebackground=blue!20,titleboxcolor = black]{remarque}{Remarque}{RemC}				%Remarque
\newboxedtheorem[boxcolor=green, background=green!5, titlebackground=green!30,titleboxcolor = black]{notation}{Notation}{NotC}				%Notation
\newboxedtheorem[boxcolor=yellow, background=yellow!0, titlebackground=yellow!30,titleboxcolor = black]{exemple}{Exemple}{ExeC}					%Exemple
\newboxedtheorem[boxcolor=magenta, background=magenta!5, titlebackground=magenta!30,titleboxcolor = black]{lemme}{Lemme}{LemC}					%Lemme


%Ensembles usuels
\renewcommand{\AA}{\mathbb{A}}
\newcommand{\CC}{\mathbb{C}}
\newcommand{\HH}{\mathbb{H}}
\newcommand{\KK}{\mathbb{K}}
\newcommand{\MM}{\mathbb{M}}
\newcommand{\NN}{\mathbb{N}}
\newcommand{\OO}{\mathbb{O}}
\newcommand{\PP}{\mathbb{P}}
\newcommand{\QQ}{\mathbb{Q}}
\newcommand{\RR}{\mathbb{R}}
\newcommand{\TT}{\mathbb{T}}
\newcommand{\ZZ}{\mathbb{Z}}

%Lettres rondes
\newcommand{\A}{\mathcal{A}}
\newcommand{\B}{\mathcal{B}}
\newcommand{\C}{\mathcal{C}}
\newcommand{\D}{\mathcal{D}}
\newcommand{\E}{\mathcal{E}}
\newcommand{\F}{\mathcal{F}}
\newcommand{\K}{\mathcal{K}}
\renewcommand{\L}{\mathcal{L}}
\newcommand{\M}{\mathcal{M}}
\newcommand{\N}{\mathcal{N}}
\renewcommand{\P}{\mathcal{P}}
\newcommand{\R}{\mathcal{R}}
\renewcommand{\S}{\mathcal{S}}
\newcommand{\T}{\mathcal{T}}
\newcommand{\V}{\mathcal{V}}
\newcommand{\W}{\mathcal{W}}
\newcommand{\X}{\mathcal{X}}

%Fleches
\newcommand{\ra}{\rightarrow}
\newcommand{\la}{\leftarrow}
\newcommand{\iso}{\stackrel{\sim}{\ra}}
\newcommand{\giso}{\stackrel{\sim}{\la}}
\newcommand{\isononcan}{\simeq}
\newcommand{\equ}{\approx}
\newcommand{\Ra}{\Rightarrow}
\newcommand{\longra}{\longrightarrow}
\newcommand{\longRa}{\Longrightarrow}
\newcommand{\La}{\Leftarrow}
\newcommand{\longla}{\longleftarrow}
\newcommand{\longLa}{\Longleftarrow}
\newcommand{\lra}{\leftrightarrow}
\newcommand{\LRa}{\Leftrightarrow}
\newcommand{\longLRa}{\Longleftrightarrow}

%Texte
\newcommand{\ssi}{si et seulement si\xspace}
\newcommand{\cad}{c'est-\`a-dire\xspace}
\newcommand{\num}{{$\mathrm{n}^{\mathrm{o}}$}}
\renewcommand{\thefootnote}{\arabic{footnote}}
\newcommand{\Turing}{\textsc{Turing }}
\newcommand{\Grzeg}{\textsc{Grzegorczyk}}
\newcommand{\Max}{\mathrm{Max}}
\renewcommand{\Pr}{\mathrm{Pr}}

%Symboles
\newcommand{\argmin}{\text{argmin}}
\newcommand{\rainbowdash}{\vdash}
\newcommand{\notrainbowdash}{\nvdash}
\newcommand{\rainbowDash}{\vDash}
\newcommand{\notrainbowDash}{\nvDash}
\newcommand{\Rainbowdash}{\Vdash}
\newcommand{\notRainbowdash}{\nVdash}
\newcommand{\bottom}{\bot}



%Du Chevalier
\newcommand{\set}[2]
{
    \ifthenelse{\equal{#2}{}}
    {
        \left\lbrace #1 \right\rbrace
    }{
        \left\lbrace #1 \:\left\vert\: #2 \vphantom{#1} \right.\right\rbrace
    }
}
\newcommand{\unionDisjointe}{\sqcup}
\newcommand{\soustractionNaturelle}{\dot{-}}
\newcommand{\iverson}[1]{\left[ #1\right]}
\newcommand{\functionArrow}{\rightarrow}
\newcommand{\surj}{\twoheadrightarrow}
\newcommand{\inj}{\hookrightarrow}
\newcommand{\bij}{\hookrightarrow \hspace{-10pt} \rightarrow}
\newcommand{\composition}{\circ}
\newcommand{\leftTuring}{\longleftarrow}
\newcommand{\rightTuring}{\longrightarrow}
\newcommand{\HALT}{\texttt{HALT}}
\newcommand{\nullTuring}{\bigcirc}
\newcommand{\UTMInit}{\diamondsuit}
\newcommand{\UTMFin}{\ddagger}
\newcommand{\kleene}[1]{#1^\star}
\newcommand{\vect}[2]
{
    \ifthenelse{\equal{#2}{}}
    {
        \mathrm{Vect}\left(#1\right)
    }{
        \mathrm{Vect}_{#1}\left(#2\right)
    }
}
\newcommand{\norme}[2]
{
    \ifthenelse{\equal{#2}{}}
    {
        \left\lVert #1 \right\rVert
    }{
        \left\lVert #1 \right\rVert_{#2}
    }
}
\newcommand{\rec}[2]{\mathrm{Rec}\left( #1, #2 \right)}
\newcommand{\RecB}[3]{\mathrm{RecB}\left( #1, #2, #3 \right)}
\newcommand{\SumB}[1]{\mathrm{SumB}\left( #1 \right)}
\newcommand{\ProdB}[1]{\mathrm{ProdB}\left( #1 \right)}
\newcommand{\MinB}[1]{\mathrm{MinB}\left( #1 \right)}
\newcommand{\Min}[1]{\mathrm{Min}\left( #1 \right)}
\newcommand{\nonDefini}{\bot}
\newcommand{\hyper}[1]{H_{#1}}
\newcommand{\knuth}[1]
{
    \ifthenelse{\equal{#1}{1}}
    {
        \uparrow
    }{
        \ifthenelse{\equal{#1}{2}}
        {
            \uparrow\uparrow
        }{
            \uparrow^{#1}
        }
    }
}
\newcommand{\et}{\wedge}
\newcommand{\biget}{\bigwedge}
\newcommand{\ou}{\vee}
\newcommand{\xor}{\oplus}
\newcommand{\non}{\neg}
\newcommand{\congru}{\equiv}
\newcommand{\inclu}{\subseteq}
\newcommand{\domine}{\prec}
\newcommand{\Ackermann}[1]{\mathcal{A}_{#1}}
\newcommand{\AckerDom}[1]{\mathfrak{T}_{#1}}
\newcommand{\alphaeq}{\leftrightarrow_\alpha}
\newcommand{\betared}{\rightarrow_\beta}
\newcommand{\betaeq}{=_\beta}
\newcommand{\etaeq}{\leftrightarrow_\eta}
\newcommand{\freeVars}[1]{freeVars\left(#1\right)}
\newcommand{\subs}[3]{\left[#1/#2\vphantom{#3}\right]#3}
\newcommand{\trans}{\ra}
\newcommand{\rtrans}{\ra}
\newcommand{\lrtrans}{\leftrightarrow}
\newcommand{\motVide}{\varepsilon}
\newcommand{\longueur}[1]{\left\lvert #1 \right\rvert}
\newcommand{\card}[1]{\left\lvert #1 \right\rvert}
\newcommand{\EulerMascheroni}{\gamma}
\newcommand{\eqFun}{\dot{=}}
\newcommand{\entiers}[2]{\left\llbracket #1, #2 \right\rrbracket}
\newcommand{\GrzegFun}[1]{\mathfrak{f}_{#1}}
\newcommand{\fst}{\pi_1}
\newcommand{\snd}{\pi_2}
\newcommand{\Elem}{\E}
\newcommand{\GrzegClass}[1]{\E^{#1}}
\newcommand{\TuringRed}{\leqslant_T}
\newcommand{\TuringEq}{\equiv_T}
\newcommand{\TuringDeg}[1]{\deg\left( #1 \right)}
\newcommand{\TuringSaut}[1]{#1'}
\newcommand{\nTuringSaut}[2]{#1^{\left( #2 \right)}}
\newcommand{\TuringEmpty}{\emptyset}
\newcommand{\MTR}{\mathfrak{M}}
\newcommand{\MTRf}{\mathfrak{M}_f}
\newcommand{\transCharniere}[2]{\mathfrak{C}\left( #1 , #2 \right)}
\newcommand{\nbOx}[1]{\mathfrak{C}\left( #1 \right)}
\newcommand{\dotsVirgule}{\ldots}
\newcommand{\dotsPlus}{\cdots}
\newcommand{\dotsLambda}{\ldots}
\newcommand{\dotsMot}{\ldots}
\newcommand{\dotsTape}{\cdots}
\newcommand{\dotsTuring}{\ldots}
\newcommand{\susp}{...}
\newcommand{\dotsProd}{\cdots}
\newcommand{\floor}[1]{\left\lfloor #1 \right\rfloor}
\newcommand{\ceil}[1]{\left\lceil #1 \right\rceil}
\renewcommand{\angle}[1]{\left\langle #1 \right\rangle}
\newcommand{\abs}[2]{\lambda #1.#2}
\newcommand{\app}[2]{#1\,#2}
\newcommand{\Space}{\texttt{[Space]}}
\newcommand{\LF}{\texttt{[LF]}}
\newcommand{\Tab}{\texttt{[Tab]}}
\newcommand{\parties}[1]{\mathcal{P}\left(#1\right)}
\newcommand{\TuringRedStrict}{<_T}
\newcommand{\Rat}[1]{\mathfrak{R}\left(#1\right)}
\newcommand{\UTMBlanc}{\texttt{\_}}
\newcommand{\dd}{\mathrm{d}}
\newcommand{\grandO}[1]{\mathcal{O}\left(#1\right)}
\newcommand{\petitO}[1]{o\left(#1\right)}
\newcommand{\opname}[1]{\operatorname{#1}}
\newcommand{\Var}[1]{\text{Var}\left( #1 \right)}
\newcommand{\prob}[1]{\PP\left( #1 \right)}
\newcommand{\esp}[1]{\EE\left[ #1 \right]}

\title{Exercices de khôlle \--- Correction}
\author{Marc \textsc{Chevalier}}
\date{2 octobre 2014}

\begin{document}
\maketitle
\setcounter{tocdepth}{2}
\tableofcontents

\section{Espaces normés}

\subsection{Un peu de norme sur les matrices}

\begin{enumerate}
    \item $\forall A\in\M_n(\CC),\norme{A}{} \geqslant 0$.
    \begin{enumerate}
        \item Soit $A=\left(a_{i,j}\right)_{\substack{1\leqslant i\leqslant n \\ 1 \leqslant j \leqslant n}}\in\M_n(\CC)$ telle que $\norme{A}{} = 0$.
        
        Comme $\forall (i,j) \in(\entiers{1}{n})^2,\vert a_{i,j}\vert \geqslant 0$, on en déduit que $\forall (i,j) \in(\entiers{1}{n})^2,\vert a_{i,j}\vert = 0$ soit $\forall (i,j) \in(\entiers{1}{n})^2, a_{i,j} \geqslant 0$ donc $A=0$.
        
        \item Soit $A=\left(a_{i,j}\right)_{\substack{1\leqslant i\leqslant n \\ 1 \leqslant j \leqslant n}}\in\M_n(\CC)$ et soit $\lambda \in \CC$.
        
        \[
            \begin{aligned}
                \norme{\lambda A}{} &= \sup\limits_{\substack{1\leqslant i\leqslant n \\ 1 \leqslant j \leqslant n}} \vert \lambda a_{i,j}\vert\\
                &= \sup\limits_{\substack{1\leqslant i\leqslant n \\ 1 \leqslant j \leqslant n}} \vert\lambda\vert\cdot\vert a_{i,j}\vert\\
                &= \vert\lambda\vert\sup\limits_{\substack{1\leqslant i\leqslant n \\ 1 \leqslant j \leqslant n}}\vert a_{i,j}\vert\\
                &=\vert \lambda \vert \norme{A}{}
            \end{aligned}
        \]
        
        \item Soit $(A,B)\in(\M_n(\CC))^2$ avec $A=\left(a_{i,j}\right)_{\substack{1\leqslant i\leqslant n \\ 1 \leqslant j \leqslant n}}$ et $B=\left(b_{i,j}\right)_{\substack{1\leqslant i\leqslant n \\ 1 \leqslant j \leqslant n}}$
        
        On a $\norme{A+B}{} = \sup\limits_{\substack{1\leqslant i\leqslant n \\ 1 \leqslant j \leqslant n}}\vert a_{i,j} +b_{i,j}\vert$
        
        Or, $\forall (i,j)\in(\entiers{1}{n})^2, \vert a_{i,j} + b_{i,j} \vert \leqslant \vert a_{i,j}\vert + \vert b_{i,j}\vert\leqslant\norme{A}{} + \norme{B}{}$.
        
        On en déduit $\norme{A+B}{} \leqslant \norme{A}{} + \norme{B}{}$.
    \end{enumerate}

    \item Soit $(A,B)\in(M_n(\CC))^2$ avec $A=(a_{i,j})_{\substack{1\leqslant i\leqslant n \\ 1 \leqslant j \leqslant n}}$ et  $B=(b_{i,j})_{\substack{1\leqslant i\leqslant n \\ 1 \leqslant j \leqslant n}}$
    
    $C=AB$. On a $C=(c_{i,j})_{\substack{1\leqslant i\leqslant n \\ 1 \leqslant j \leqslant n}}$ avec $\forall (i,j) \in(\llbracket 1,n \rrbracket)^2,c_{i,j}=\sum\limits_{k=1}^na_{i,k}b_{k,j}$.
    
    Donc $\forall (i,j) \in(\llbracket 1,n \rrbracket)^2,\vert c_{i,j} \vert \leqslant \sum\limits_{k=1}^n\vert a_{i,k}\vert\cdot\vert b_{k,j}\vert \leqslant \sum\limits_{k=1}^n \Vert A \Vert\, \Vert B \Vert = n \Vert A \Vert\, \Vert B \Vert$.

    \item Pour tout entier naturel $p \geqslant 1$, notons $(P_p)$ la propriété : $\norme{A^p}{} \leqslant n^{p-1} \Vert A \Vert^p$.
    Prouvons que $(P_p)$ est vraie par récurrence.
    
Pour $p = 1$ : $\Vert A^1\Vert = n^0\Vert A \Vert^1$ donc $(P_1)$ est vraie.

Supposons la propriété $(P_p)$ vraie pour un rang $p \geqslant 1$, c'est-à-dire $\Vert A^p\Vert \leqslant n^{p-1} \Vert A \Vert^p$.

Prouvons que $(P_{p+1})$ est vraie. 

$A_{p+1} = \Vert A \times A^p \Vert$ donc, $\Vert A^{p+1} \Vert \leqslant n \Vert A \Vert \, \Vert A^p\Vert$.

Alors, en utilisant l’hypothèse de récurrence,

$\Vert A^{p+1}\Vert \leqslant n \Vert A \Vert n^{p-1} \Vert A \Vert^p = n^p\Vert A\Vert^p$

On en déduit que $(P_{p+1})$ est vraie.

    \item On a $\forall p\in\NN^*,\left\Vert \frac{A^p}{p!}\right\Vert \leqslant \frac{1}{n} \frac{(n\Vert A \Vert)^p}{p!}$.
    
    Or $\forall x\in\RR$, la série exponentielle $\sum \frac{x^p}{p!}$ converge, donc $\sum \frac{(n\Vert A \Vert)^p}{p!}$ converge.
    
    Donc, par comparaison de séries à termes positifs, la série $\sum\frac{A^p}{p!}$ est absolument convergente . Or$\M_n(\CC)$ est de dimension finie, donc $\sum\frac{A^p}{p!}$ converge.
\end{enumerate}

\subsection{Norme de fonction \texorpdfstring{$(\Delta)$}{Delta}}

\begin{enumerate}
    \item La seule propriété qui pose problème est de prouver que si $N_g(f)=0$ alors $f=0$. Si $N_g$ n'est pas une norme, alors il existe $f\in\C([0,1]), f\neq 0$ avec $N_g(f)=0$. Autrement, $f(x)g(x) = 0$ pour tout $x\in[0,1]$. Puisque $f$ est continue et non nulle, il existe un intervalle $I$, non réduit à un point, sur lequel $f$ ne s'annule pas. Mais alors, on en déduit que $g$ doit être nulle sur $I$. Réciproquement, si $g$ s'annule sur un intervalle non réduit à un point, alors on peut construire $f$ continue qui s'annule hors de $I$ et tel qu'il existe $a \in I$ avec $f(a) \neq 0$ (faire un dessin et construire $f$ comme un \og pic \fg{} ). On a donc $f \neq 0$ et $N_g(f) = 0$, donc $N_g$ n'est pas une norme. Par contraposée, on en déduit que $N_g$ est une norme si et seulement si $g$ ne s'annule pas sur un intervalle non réduit à un point.
    
    \item Remarquons déjà que $g$, continue sur le segment $[0,1]$, est bornée par une constante $M > 0$. On a donc $N_g(f) \leqslant M \Vert f \Vert_\infty$ pour tout $f \in E$. Supposons de plus que $g$ ne s'annule pas. Alors, puisque $|g|$ est continue et atteint ses bornes sur $[0,1]$, il existe $\delta >0$ tel que $|g(x)| \geqslant \delta$ pour tout $x\in [0,1]$. On a alors clairement $N_g(f)\geqslant \delta \Vert f \Vert_\infty$ et les deux normes sont équivalentes.

Réciproquement, si $g$ s'annule, prouvons que les deux normes ne sont pas équivalentes. Soit $M > 0$. On va construire $f\in E, f\neq 0$, tel que $\Vert f \Vert_\infty \geqslant MN_g(f)$. Pour cela, on sait, par continuité de $g$, qu'il existe un intervalle $I$, non-réduit à un point, et contenu dans $[0,1]$, tel que $|g(x)|\leqslant \frac{1}{M}$ pour tout $x \in [0,1]$. Comme à la question précédente, on peut construire $f$ nulle en dehors de $I$, avec $\Vert f \Vert_\infty \leqslant 1$ et $f(a) = 1$pour au moins un $a$ de $I$. On a alors $\Vert f \Vert_\infty = 1$ tandis que $N_g(f) = \sup\limits_{x\in I} \vert g(x)f(x)\vert \leqslant \frac{1}{M}$. Ceci prouve bien l'inégalité annoncée, et les deux normes ne sont pas équivalentes. En conclusion, on a démontré que les deux normes sont équivalentes si et seulement si $g$ ne s'annule pas.
\end{enumerate}

\subsection{Fermeture des sous-espace d'un evn de dimension finie \texorpdfstring{$(\Delta)$}{Delta}}
Soit $F$ un tel sous-espace et $(e_1,\ldots,e_p)$ une base de $F$. On complète $(e_1,\ldots,e_p)$ en $(e_1,\ldots,e_q)$ une base de $E$. On considère la norme sur $E$ :
$$
    \begin{aligned}
        N:E&\ra \RR^+\\
        \sum_{i=1}^q x_ie_i &\mapsto \max_i\vert x_i\vert
    \end{aligned}
$$
Rappelons que, puisque $E$ est de dimension finie, toutes les normes sur $E$ sont équivalentes, il suffit de prouver que $F$ est fermé relativement à cette norme. Soit $(x(n))$ une suite de $F$, qui converge avec $x\in E$ pour $N$. Chaque $x(n)$ s'écrit $x(n)=x_1(n)e_1+\ldots+x_q(n)e_q$ avec $x_i(n) =0 si i\geqslant p+1$. On fait la même décomposition pour $x$.

Remarquons maintenant que $\vert x_i(n)-x_i\vert \leqslant N(x(n)-x)$. Ceci prouve que chaque suite $(x_i(n))$ converge vers $x_i$. En particulier pour $i \leqslant p+1 ,x_i=0$ ce qui prouve que $x\in F$.

\subsection{L'intégrale jamais nulle}
$E=\RR_n[X]$. On pose pour $P=\sum\limits_{i=0}^n a_i X_i$ 
$$
    \begin{aligned}
        \norme{P}{1}&=\int_0^1\vert P(t) \vert \dd t\\
        N(P) &= \sup_k \vert a_k\vert
    \end{aligned}
$$
Ce sont des normes. Grâce à $N$ on remarque facilement que $E_n$ est un fermé de $E$.

Si $\inf\limits_{P\in E_n} \int_0^1\vert P(t)\vert \dd t = \inf\limits_{P\in E_n}\norme{P}{1}=0$, on peut trouver une suite $(P_k)_{k\in\NN}$ de $E_n$ telle que  $\norme{P_k}{1}\ra 0$. Autrement dit $(P_k)$ converge vers 0. Mais puisque $E_n$ est fermé, on aurait $0\in E_n$, ce qui n'est pas le cas.

\section{Connexité par arcs}

\subsection{Échauffement}

\begin{itemize}
    \item Le cas $d=1$ est trivialement non connexe.
    
        \bigskip    
    
        On suppose $d\geqslant 2$. On prend $(a,b) \in \RR^d\setminus \QQ^d$. $a=(a_1,\ldots,a_d)$ et $b=(b_1,\ldots,b_d)$.
        On suppose que $a_{i_0} \in \RR\setminus \QQ$ et $b_{j_0} \in \RR\setminus\QQ$.
        
        On construit un chemin continu entre $a$ et $b$ tel que tous les points du chemin ont au moins une coordonnée irrationnelle.
        \[
            (a_1,\ldots,a_{i_0-1},a_{i_0},a_{i_0+1},\ldots,a_d) \longra (b_1,\ldots,b_{i_0-1},a_{i_0},b_{i_0+1},\ldots,b_d) 
        \]
        la $i_0^\text{ème}$ composante est irrationnelle.
        
        Si $i_0 \neq j_0$ : 
        \[
            (b_1,\ldots,b_{i_0-1},a_{i_0},b_{i_0+1},\ldots,b_{j_0},\ldots,b_d) \longra (b_1,\ldots,b_{i_0-1},b_{i_0},b_{i_0+1},\ldots, b_{j_0},\ldots,b_d)
        \]
        la $j_0^\text{ème}$ composante est irrationnelle.

        Sinon, on prend $k_0 \neq i_0$ et $x\in\RR\setminus\QQ$.
        \[
            \begin{aligned}
                (b_1,\ldots,b_{i_0-1},a_{i_0},b_{i_0+1},\ldots,b_{k_0},\ldots,b_d) &\longra (b_1,\ldots,b_{i_0-1},a_{i_0},b_{i_0+1},\ldots, \underset{\underset{\text{rang }k_0}{\uparrow}}{x},\ldots,b_d)\\
                (b_1,\ldots,b_{i_0-1},a_{i_0},b_{i_0+1},\ldots, x,\ldots,b_d) &\longra (b_1,\ldots,b_{i_0-1},b_{i_0},b_{i_0+1},\ldots, x,\ldots,b_d) \\
                (b_1,\ldots,b_{i_0-1},b_{i_0},b_{i_0+1},\ldots, x,\ldots,b_d) &\longra (b_1,\ldots,b_{i_0-1},b_{i_0},b_{i_0+1},\ldots, b_{k_0},\ldots,b_d)
            \end{aligned}
        \]
        où, respectivement, les composantes d'indice $i_0$, $k_0$ et $i_0$ sont irrationnelles.

    \item Soit $d\in\NN^*$.
    
        On prend deux points $(a,b) \in (\RR\setminus\QQ)^d$ et un chemin continu
        \[
            \begin{aligned}
                \varphi : [0,1] &\functionArrow \RR^d\\
                t &\mapsto (\varphi_i(t))_{i\in\entiers{1}{d}}
            \end{aligned}                    
        \]
        tel que $\varphi(0) = a$ et $\varphi(0) = b$.
        
        On suppose $a\neq b$. On prend $i_0$ tel que $a_{i_0} \neq b_{i_0}$.
        
        $\forall i \in \entiers{1}{d}, \varphi_i \in \C^0([0,1], \RR)$.
        
        $\varphi_{i_0}$ est une application continue de $[0,1]$ dans $\RR$ avec $\varphi(0) \neq \varphi(1)$. Comme $\QQ$ est dense dans $\RR$, il existe $t_0$ tel que $\varphi(t_0) \in \QQ$. Par conséquent, pour tout $d \in \NN^*, (\RR\setminus\QQ)^d$ n'est pas connexe par arc.
        
    \item Cet espace est celui des points dont au moins une coordonnée est rationnelle.
    
        Si $d=1$, cet espace est trivialement non connexe par arc.
    
        Soit $a = (a_1,\ldots,a_d)$ un point de cet espace. On prend $i_0$ tel que $a_{i_0} \in \QQ$.
        
        \[
            \begin{aligned}
                (a_1,\ldots,a_{i_0-1},a_{i_0},a_{i_0+1},\ldots,a_d) &\longra (0,\ldots,0,a_{i_0},0,\ldots,0)\\
                (0,\ldots,0,a_{i_0},0,\ldots,0) &\longra (0,\ldots,0,0,0,\ldots,0)
            \end{aligned}
        \]
        
        Donc on peut arriver par chemin continu à $0$ à partir de tout point. Par conséquent, $\RR^d\setminus(\RR^d\setminus \QQ^d)$ est connexe par arc si $d\geqslant 2$.
\end{itemize}

\subsection{Connexe, mais pas par arcs}

    $\overline{\Gamma} = \Gamma \cup (\set{\lbrace 0 \rbrace \times [-1,1]}{})$ est connexe, comme adhérence d'un connexe puisque graphe d'une fonction continue sur un intervalle réel. Mais il est trivialement pas connexe par arc.

\section{Topologie}

\subsection{Convexité de l'adhérence et de l'intérieur d'un convexe}
Soit $(x,y)\in C$, et $t\in [0,1]$. $x$ (resp. $y$) est limite d'une suite $(x_n)$ (resp. $(y_n)$) d'éléments de $C$. Puisque $C$ est convexe, la suite $z_n=tx_n+ (1-t)y_n$ est dans $C$. On passe à la limite : la suite $(z_n)$ converge vers $tx + (1-t)y$, et cette limite est dans $\overline{C}$. D'où $tx+ (1-t)y \in \overline{C}$, ensemble qui est donc convexe.

Prouvons maintenant le résultat concernant l'intérieur. Soit $(x,y)\in \overset{\circ}{C},x\neq y$, et soit $z\in ]x,y[$. Alors il existe une (unique) homothétie de centre $x$ qui envoie $y$ sur $z$ (une homothétie de centre $x$ est une application de la forme $w \mapsto x+\lambda (w - x)$. Cette homothétie transforme la boule de centre $y$ et de rayon $\delta$ en la boule de centre $z$ et de rayon $\lambda \delta$. Soit $\delta >0$ tel que $B(y,\delta) \subseteq C$ et soit $w \in B(z,\lambda \delta)$. Alors $w=h(u)$, avec $u$ un point de $B(y,\delta)$, et $h$ l'homothétie précédemment considérée. En particulier, $w$ est sur le segment $[x,u]$ et est donc un élément de $C$. Autrement dit, on vient de prouver que $B(z,\lambda \delta) \subseteq C$, ce qui prouve $z \in \overset{\circ}{C}$. $\overset{\circ}{C}$ est convexe.

\subsection{Adhérence des sous-espaces vectoriels}
\begin{enumerate}
    \item Soit $(x,y) \in \overline{V}^2$, et $(\lambda,\mu) \in \RR^2$. $x$ (resp. $y$) est limite d'une suite $(x_n)$ (resp. $(y_n)$) d'éléments de $V$ Puisque $V$ est un sous-espace vectoriel, la suite $z_n = \lambda x_n + \mu y_n$ évolue dans $V$. On passe à la limite : $z_n$ converge vers $z=\lambda x + \mu y$ qui est un élément de $\overline{V}$ puisque $z_n \in V$.
    \item Soit $a\in V$ et $\varepsilon > 0$ tel que $V(a,\varepsilon) \subseteq V$. Soit $x\in B(0,\varepsilon)$. Puisque $x+a\in B(a,\varepsilon) \subseteq V$ et que $V$ est un espace vectoriel, on a $x \in V$. D'où $B(0,\varepsilon)\subseteq V$. Si maintenant $x\neq 0$ est dans $E$, alors $a=\frac{\varepsilon x }{2 \Vert x \Vert}$ est dans $B(0,\varepsilon)$ donc dans $V$, et puisque $V$ est un sous espace vectoriel, c'est aussi le cas de $x$.
\end{enumerate}

\section{Application linéaire continue}

\subsection{Régularité de la dérivation \texorpdfstring{$(\Delta)$}{Delta}}
Pour $a\in\RR$, la fonction $f_a(x)=e^{ax}$ est dans $E$, et elle vérifie $Df_a=af_a$. Or si $D$ était continue pour la norme $N$, il existerait une constante $C>0$ telle que $\forall a\in \RR, N(D(f_a)) \leqslant CN(f_a)$. On obtiendrait alors que, pour tout $a\in\RR$, $\vert a \vert N(f_a) \leqslant CN(f_a) \Rightarrow \vert a \vert \leqslant C$. D'où une contradiction.

\end{document}